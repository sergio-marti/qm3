\normalsize
\subsection[stats]{stats.py}
Simple statistical analysis of mono-dimensional sets of data, arranged in different tools: statistics (\func{stats},
\func{autocorrelation}, \func{sampling\_ratio}), clustering (\func{k\_means}) and principal component analysis (\func{PCA}).

\begin{pyglist}[language=python,fvset={frame=single}]
def stats( x )

def autocorrelation( x, k = 1 )

# The value of the sampling ratio that arises from any given data sequence is the factor 
# by which the number of configurations sampled must be increased in order to obtain the
# same precision that would result from randomly distributed data points.
def sampling_ratio( x )

def k_means( x, k )

class PCA( x )
    def select( sel, reduced = True )
\end{pyglist}

\footnotesize
\begin{pyglist}[language=python,fvset={frame=single}]
>>> import qm3.maths.stats as stats
>>> import qm3.maths.matrix as matrix

>>> stats.stats( [ 0.388, 0.389, 0.410 ] )
(0.39566666666666667, 0.01242309676905613)

>>> stats.autocorrelation( range( 10 ) )
0.7

>>> stats.autocorrelation( range( 10 ), 2 )
0.412121212121

>>> stats.sampling_ratio( range( 10 ) )
5.66666666667

>>> stats.k_means( [ 7, 4, 10, 16, 13, 7, 3, 5, 7, 3, 13, 14, 12, 11, 10, 7, 7, 5, 3, 3 ], 2 )[0]
{0: [10, 16, 13, 13, 14, 12, 11, 10], 1: [7, 4, 7, 3, 5, 7, 3, 7, 7, 5, 3, 3]}

>>> x = [ [ 7, 4, 10, 16, 13 ], [ 7, 7, 5, 3, 3 ], [ 7, 3, 5, 7, 3 ], [ 13, 14, 12, 11, 10 ] ]
>>> matrix.mprint( sum( x, [] ), 4, 5 )
4 rows x 5 columns
     7.000     4.000    10.000    16.000    13.000
     7.000     7.000     5.000     3.000     3.000
     7.000     3.000     5.000     7.000     3.000
    13.000    14.000    12.000    11.000    10.000

>>> o = stats.PCA( x )
>>> o.val
[22.849549071130916, 1.8559979853574662e-16, 3.3744840968681555, 0.17596683200093394]

>>> matrix.mprint( o.select( [ 0,2,3 ] ), 3, 5 )
3 rows x 5 columns
     6.569     3.204    10.000    16.525    13.702
     7.443     3.739     5.000     6.484     2.333
    11.492    12.470    12.000    12.466    11.572

>>> matrix.mprint( o.select( [ 0,2,3 ], False ), 4, 5 )
4 rows x 5 columns
     7.000     4.000    10.000    16.000    13.000
     7.000     7.000     5.000     3.000     3.000
     7.000     3.000     5.000     7.000     3.000
    13.000    14.000    12.000    11.000    10.000

>>> import numpy
>>> print( qm3.maths.stats.npk_means( numpy.array(
         [ 7, 4, 10, 16, 13, 7, 3, 5, 7, 3, 13, 14, 12, 11, 10, 7, 7, 5, 3, 3 ] ), 2 )[0] )
{0: array([10, 16, 13, 13, 14, 12, 11, 10]), 1: array([7, 4, 7, 3, 5, 7, 3, 7, 7, 5, 3, 3])}

>>> x = numpy.array( [ 7, 4, 10, 16, 13, 7, 7, 5, 3, 3, 7, 3, 5, 7, 3, 13, 14, 12, 11, 10 ] ).reshape( (4,5) )
>>> o = qm3.maths.stats.npPCA( x )
>>> o.val
[1.26074783e-15 1.75966832e-01 3.37448410e+00 2.28495491e+01]

>>> o.select( [ 1,2,3 ] )
[[-0.50789481  0.47039089  0.          0.4659131  -0.42840918]
 [ 2.44311595 -1.26087207  0.          1.48432454 -2.66656843]
 [ 3.43121368  6.79624411  0.         -6.52531268 -3.70214511]]

>>> o.select( [ 1,2,3 ], False )
[[ 7.  4. 10. 16. 13.]
 [ 7.  7.  5.  3.  3.]
 [ 7.  3.  5.  7.  3.]
 [13. 14. 12. 11. 10.]]
\end{pyglist}
